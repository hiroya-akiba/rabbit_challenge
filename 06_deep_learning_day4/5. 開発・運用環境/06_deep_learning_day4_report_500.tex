\documentclass{ltjsarticle}
\usepackage{amsmath}
\usepackage{amssymb}
\usepackage{ascmac}
\usepackage[dvipdfmx]{graphicx}
\usepackage{tabularx}
\usepackage[colorlinks=true, allcolors=blue]{hyperref}
\usepackage{fancybox}
\usepackage{tikz}
\usepackage{subcaption}
\usetikzlibrary{shapes,arrows}

\begin{document}

\title{500. 開発・運用環境}
\author{秋葉洋哉}
\maketitle

\section{Docker}
\subsection{概要}
仮想環境とは、ハードウェア上で独立した複数の環境を実行する技術で、ホスト型、ハイパーバイザー型、コンテナ型の3つに分類できる。それぞれは、リソースをどのように共有するかに違いがある。
\begin{itemize}
  \item ホスト型: ホストOSの上で仮想環境を実行する。
  \item ハイパーバイザー型: ハードウェア上で直接実行する。
  \item コンテナ型: ホストOSのカーネルを共有して実行する。
\end{itemize}
Dockerは、コンテナ型の仮想環境を提供するツールであり、コンテナ型の仮想環境を構築・運用するためのプラットフォームである。
\par
Dockerとは、アプリケーションと依存関係をコンテナとしてパッケージ化したもので、アプリケーションの依存関係をインフラから分離することで、アプリケーションの開発、テスト、デプロイを簡素化することができる仮想環境、と定義できる。
\par
Dockerは、エンジン・イメージ・コンテナの3つの要素で構成されている。エンジンは、Dockerの実行環境であり、イメージは、コンテナの実行に必要なファイルや設定をまとめたものであり、コンテナは、イメージを実行したものである。

\subsection{Dockerの活用}
Dockerは、以下のような活用方法がある。
\begin{itemize}
  \item アプリケーションの開発とテスト: 現代の開発の複雑さを軽減する
  \item 本番環境でのデプロイ: マシン毎の環境差異を解消する
  \item マイクロサービスアーキテクチャの実装: システムの柔軟性と耐障害性を向上する
  \item 環境の統一と再現性の確保: 開発、テスト、本番環境の動作を統一する
\end{itemize}

\subsection{Dockerのコマンド}
Dockerのコマンドは、以下のようなものがある。
\begin{itemize}
  \item docker run: コンテナを起動する
  \item docker ps: 実行中のコンテナを一覧表示する
  \item docker images: ローカルに保存されたイメージを表示する
  \item docker build: Dockerfileを基にイメージを作成する
  \item docker pull: イメージを取得する
  \item docker push: イメージをリポジトリにプッシュする
  \item docker start: 停止しているコンテナの起動
  \item docker stop: 実行中のコンテナを停止する
  \item docker rm: コンテナを削除する
\end{itemize}

\subsection{Dockerfileの主要な記述}
Dockerfileは、Dockerイメージを構築するためのスクリプトであり、以下のような記述がある。
\begin{itemize}
  \item FROM: ベースとなるイメージを指定する [例: FROM ubuntu:20.04]
  \item RUN: コマンドを実行する [例: RUN apt-get update\&\&apt-get install -y vim]
  \item COPY: ファイルやディレクトリをコピーする [例: COPY ./app /app]
  \item CMD: コンテナが起動した際に実行するコマンドを指定する [例: CMD ["python", "app.py"]]
  \item WORKDIR: 作業ディレクトリを指定する[例: WORKDIR /app]
  \item EXPOSE: ポートを公開する
  \item ENTRYPOINT: コンテナが起動した際に実行するコマンドを指定する [例: ENTRYPOINT ["echo"]]
  \item ADD: ファイルやディレクトリをコピーする [例: ADD https://example.com/file.txt /file.txt]
  \item ENV: 環境変数を設定する [例: ENV MY\_ENV\_VARIABLE=value]
\end{itemize}
.dockerignoreファイルは、Dockerイメージをビルドする際に無視するファイルやディレクトリを指定するファイルである。
可能な限り、公式のイメージを利用することが推奨される。

\subsection{Dockerを用いた深層学習モデル開発}
Dockerは、GPUを利用するための機能を提供している。NVIDIA Container Toolkitを利用することで、DockerコンテナからGPUを利用することができる。深層学習モデルの開発において、Dockerを使用することで、GPUだけでなく、データセットやモデルのバージョン管理を行うことができる。

\subsection{コンテナオーケストレーション}
コンテナオーケストレーションとは、コンテナのデプロイ、スケーリング、運用の自動化プロセスであり、大規模アプリケーションや、マイクロサービスの効果的な管理を可能にする。代表的なコンテナオーケストレーションツールとして、Kubernetes(オープンソース・大規模運用向け)や、docker-compose(開発環境・小規模デプロイ向け)といったものがある。
\par
主な機能としては、以下のようなものがある。
\begin{itemize}
  \item サービスディスカバリーとロードバランシング: DNS名やIPアドレスを使って、コンテナを外部に公開する
  \item ストレージオーケストレーション: 指定したストレージをコンテナにマウントする
  \item 自動ロールアウト・ロールバック: コンテナの状態を自動的に管理する
  \item セルフヒーリング: 障害が発生したコンテナを自動的に修復する
\end{itemize}

\end{document}